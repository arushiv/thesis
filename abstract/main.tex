\ac{T2D} is a complex disease that affects an estimated 415 million people worldwide. \ac{GWAS} have identified $>$240 genetic signals that encode predisposition to this disease and related traits. However, the underlying biological mechanisms driving this predisposition are largely unknown, which is a serious impediment in designing precision therapeutic strategies. The focus of my research is to untangle the genetic complexity of \ac{T2D} to better understand the biological mechanisms of how disease predisposition is encoded in our DNA. Specifically, I aim to understand how \ac{T2D} genetic risk variants modulate gene expression in orchestrating disease mechanisms. 


I utilize high throughput molecular profiling data in human pancreatic islets and other diverse tissues along with human and rodent cell line model systems and employ computational and experimental approaches to map functional signatures of genetic variants associated with T2D. First, I compared gene regulatory annotations defined using diverse epigenomic data across 4 cell types to compare their cell specificities and genetics of gene expression regulation. I observed that genetic variants in genomic regions with more cell type-specific enhancer chromatin have lower effects on gene expression than variants in genomic regions with more ubiquitous promoter chromatin. However, genetic variants in cell type-specific enhancer regions have higher effects in chromatin accessibility than those in less cell type-specific promoter regions. Second, I integrated GWAS data with various -omics data in islets to nominate biological mechanisms. I observed that T2D risk variants confluently disrupt DNA binding motifs of the \ac{TF} \ac{RFX} in accessible regions. Third, I describe large scale \ac{eQTL} mapping efforts along with integration of epigenomic data to describe molecular regulatory mechanisms. Utilizing such large eQTL and integrating information such as chromatin accessibility and TF binding predictions helped elucidate \textit{in vivo} TF activity preferences. Fourth, I describe profiling and analysis of the enhancer transcriptome in islets, which I then integrate with other available epigenomic data to better understand the characteristics of gene regulation. 